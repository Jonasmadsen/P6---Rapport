Supervisor meeting 10/3 13:14

Attendies:

Kristian V
Kristian O
Jonas
Jeppe

Tung

Location: Meeting was held in group room

We read the paper provided by tung by not sure we completely understand it. We understood the overall design and the theroy of what they wanted to achieve.
Fortuneatly they have the source code on github.

Kristian says they dont use label data in the report, but use herustic instead.
Tung says that we need label anyways for what were doing. Tung knows that the paper is hard becuase it is from the top research. Tung says we will try to learn a easier paper first in order to help us understand the hard paper.

We should check "autoencoder". Is the paper not about autoencoders? tung says no its not, the paper is from a journal called science. We should read/see tutorials on autoencoder along with autoencoder anamoli detection. Thuey talk about how to find this paper. 

"Outlier detection using neroul network etc."

We read the tutorial for autoencoder and this paper and a other paper called "variancele autoencoder". Genrellay stuff by Dedrick p kingma.
Autoencoder variation something. 

Tung says that paper is a reasonable paper, they are simmerlar.

One is neural network based the other is statistical based. After we read them we can learn stochastic sequence modelling. 

Jeppe downloads a lot of papers that will help. The main paper is very advanced.

Do you want to see progress on report? Tung says yes we should send it to him he will try to fix it. He says we must read the papers. We consider reading these papers part of designing the Outlier detection module. The paper must follow the report??

We have sequence data and non sequence data. Temporal data is sequence data, if we use autoencoder we cannot model the temporal data. 

www. 17 seasonal kbi is another paper with anomali detection outlier detection things. This paper uses a variantel autoencoder proposed by Kingma.
First they transform the time series into "non sequence data" they then feed the data to autoencoder. We can read and compare this paper. 

Tung says tensorflow is okay but maybe consider ?? other pytonthing.
We also could use deep learning for java. It is best we continue to use tensorflow since the group have expenrience with that already.

We will send the report as it is know and then we will being reading. Tung will send us a small summary of all the papers we should read.