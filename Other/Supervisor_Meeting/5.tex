Supervisor meeting 30/3 12:55

Attendies:

Kristian V
Kristian O
Jonas
Jeppe
Rasmus

Tung

Location: Meeting was held on discord due to COVID 19.

Tung was late by 2 minute for the first time ever, historic event.

Tung opens agenda. He sent us a small dataset asks us if we saw it. Yes we did. He said we should only use the first 2. A1 and A2.

The visualatation is to visualize the output of the algorithm. We delete A3 and A4 we dont need them.

After we finish the algortihm tung will send bigger dataset.

Tung read the report.

2014  is too old, he sent us a never paper and apprently a lot has happended since. The model for anomaly detection is last year. However we use the "old" paper as refference for how VAE work in generel.

We can propose two different models and combat them. If we choose the kdd paper that is enough, if we choose the simpler paper, we must compare between them to make the paper more complex. We will focus on kdd model.

The tool we use for creating figures is draw.io
Tung thinks we should start to impliment the KDD model in sprint.
The database model is still missing. And should be in sprint 2.

We also want to describe KDD paper, we should describe it as soon as possble becuase it is main part of Database structure can be added later, but kdd paper is needed now.
We lack a lot of the model. He expects that we need atleast about 10 pages on the kdd model.

10 pages on kdd model.

In sprint 3 we will run experiement. We must hurry as we are a bit behind. Tung didnt see many changes in the report. 

Tung asks if we have problems working from home. Kristian says yes the productivity is recused also because we cant share the knowledge.

Tung says we can use skype to do pair programming. But it is definatly harder from home. Also it would be easier if we were together. Also the group is split in knowledge since we havent all had MM course.

Tung says the matehmatical and probalistic model is not important since NN is like a blackbox. We need to have atleast a prototype. 

We will work toward a prototype. 

Tung says we dont need to understand everything to start implimenting it. Se we should just go for it.

Tung he didnt see the formal definitions, kristian says it is in 1.3

Tung says title is still bad, we agree the title is not priority right now.

Tung gives tip: we must have evidence for 1.3 and we can do that with citation, all centences must have evidence. 

The online tutorial online does not have high confidence, we change the tutorial to a scietific paper to give better confidence.

Tung is quite worried about the slow progress so we should ask him between meetings. 

We add some example method to give overwiev of the whole project/code. We dont need to write code detials for the prototype, but more overview, like for each class the mothods and fields. 


TUNGs idea of a prototype: Prototype of the coding---(UML: use case, class diagram, prototype class)

Tung expects that finishing the code for the prototype is too much so 50-70 percent of the implementation is okay.

We can copy from the kdd paper github but he expects we will understand about that much.

We will try to copy github kdd paper, and tweak.

Tung says the github of kdd paper is very complicated. Becuase the library they use is a custom library, they use tensorflow but on top of the library they use tensorflow-"nippet" and it is custom so it is hard to understand what is going on.

We might be able to write it better ourselves and we are guaranteed to understand it better.

But it is our choice. We can think about it.
We have checked it but in depth yet.

Tung read the github and he needs 3 days to make it work and understand, but he already has the understanding so for us it might be so much longer.
And their tensorflow version is also quite old.
This paper is also created by a guy that works for alibaba.

TS snippet, and tung says it is hard to understand. We impliment a simple version in tensorflow. 

We can send TUNG a decission of it we want to impliment one or the other or copy one of the other.

Tung says we will figure 3b of paper. THe CIU is implimented allready in tensorflow that is easy paper and it appears clear but he doesnt know why they complicate everything with  tfsnippet.

For tung it is easy. Kristian says he thinks we can manage it, tung says we can send him email.
Since we also missed the deadline last we need to speed up everything.


We wanted to ask about how much documentation we should have regarding charts. Do we need to describe how the API documentation. Tung says its not as important as how we get along with the KDD paper. The interface can wait till sprint 3 and 4.

We dont have task prioroty only risk management, the task mangement can we add later, but for now we should think about what tasks we should do first. 

For know we need to focus on our group and not focus on documenting interface choice. '

Tung out! Good luck with the implimentation see you.