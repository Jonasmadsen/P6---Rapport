Supervisor meeting 4. May

Attendies:

Kristian V
Kristian O
Jonas
Jeppe
Rasmus

Tung

Location: Meeting was held on discord due to COVID 19.

Kristian deler skærm. Tung says we must normalize. We say that we do that while training. Tung says we must do that while. We fix the x and y axis (normalize)

The datasets are too small. Tung will give us tip so we dont spend many time oon training. Early stopping is a strategy we can use. See in chat.

1. the loss value of current epoch - the loss value of previous epoch < epsilon ==> we break the training

loss[t] - loss[t-1] < eps
  break
2. the loss value of current epoch does not change from 3 previous epochs ==> we break the training

3. The mean of three loss values less than eps

( loss[t] + loss[t-1] +  loss[t-2])/3 < eps
  break
  
eps might be = 1e-7

Our loss varies a lot shows chart.

Tung says we can use variable learning rate. For each epoch we multiply the learning rate by 0.8.

We can play around with it!

If we make the demo paper we should put it in the report. It will be a benefit to the paper.

Can we use multiple datasets. Concatination vs Batch training.
Tung says both will work, and it will make model stronger, if we wanna make a translator from english to french we can put more stories into we must use something with the same distrobution. 
Tungs tip for training with multiple datasets. We can cluster the data from 100 time series to five groups of simmerlar data then we can create 5 models.


Should we justify the simplifications we make (like time pressure..)?
We reffer to the KDD and we propose some similar model, so we use only the idea.

We propose something to follow the stream of using variational ... bla bla... 

We can write like that in the report but we cannot write like that in the report. We need to propose a new model. In the paper we need be careful.
