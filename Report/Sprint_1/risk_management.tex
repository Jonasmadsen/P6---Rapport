\section{Risk Management}
In this section, we present a risk management strategy. Risks will be identified and evaluated for comparison. For each risk, we provide a risk management strategy. We provide individual management strategies in a \Gls{rmmm}. Appendix \ref{app:B} contains the \Gls{rmmm}.

\subsection{Risk Identification}
We identify risks in two categories relevant to our project. The categories are product-specific risks and generic risks. Generic risks are known and predictable, given an arbitrary project. We base generic risks on some generic subcategories of risks applicable to every software project \cite[p.~747]{a_practitioners_approach}. Appendix \ref{app:A} illustrates both categories of risks we identified as an item checklist. In the following risk analysis, we evaluate each item on the list according to the likelihood of occurrence and consequence.


\subsection{Risk Analysis}
We assess the identified risks according to their probability (\textit{P}) of occurring, the loss (\textit{L}) of work hours, and the exposure (\textit{E}), i.e., the expected average loss of a risk. For a given risk \textit{r}, the risk's exposure \textit{E(r)} are calculated as a product of the risk's probability of occurring \textit{P(r)} and the loss \textit{L(r)}. We can only assess a given risk by evaluating both its probability and consequence, i.e., loss of work hours. Based on our knowledge of software engineering, we assess the probability of each risk. We measure the loss in lost work hours per group member to manage the risk's consequence. The loss and consequence of risk can also affect other factors than time. However, there exists no straight forward procedure when assessing a risk's consequences. Therefore, since this project has a hard deadline within four months, time will be our main focus of loss. 
\newline

\subsection{Risk Mitigation, Monitoring and Management Plan}
Appendix \ref{app:B} contains a prioritized list of risks and their plans for mitigation, monitoring, and management. Based on the list of risks and their exposure, we establish a cutoff. That is a level of exposure that is acceptable without any further strategy of risk management. The risks below this cutoff are not worth spending more time on, compared to spending the effort elsewhere. For risks above the cutoff line, we develop mitigation, monitoring, and management plans\newline 
In order to mitigate each risk, we consider risk avoidance and risk minimization. How we can reduce a risk's probability and impact should it occur, forms our mitigation plan. We also monitor each risk throughout the project. Monitoring each risk allows for management of each risk should it progress towards occurring. We finally develop a management plan for each risk, should it occur.