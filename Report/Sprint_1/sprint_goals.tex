\section{Sprint Goals}
The main sprint goal is to analyze \gls{astep} and gain an understanding of what \gls{astep}-2019 contains. Based on this analysis, we determine which parts of aSTEP-2019 we can reuse and redesign for aSTEP-2020, and which parts we have to develop. We also have to agree on the initial design of \gls{ui} and documentation, with the other groups in the aSTEP-2020 project.\newline 

\bgroup
\def\arraystretch{1.8}
\begin{table}[htbp]
    \centering
    \begin{tabular}{| m{7cm} |}
        \hline
        \multicolumn{1}{|>{\centering\arraybackslash}m{70mm}|}{\textbf{Goals}} \\
        \hline
        Read and understand legacy code. \\
        \hline
        Determine which parts of the aSTEP-2019 project can be reused or redesigned. \\
        \hline
        Determine what needs to be added to fulfill the goals of aSTEP-2020. \\
        \hline
        Agree on the initial design of \gls{ui} and documentation. \\
        \hline
        Conduct a risk analysis on our project. \\
        \hline
    \end{tabular}
    \caption{The first sprint goals.}
    \label{tab:sprint_1_goals}
\end{table}
\egroup

\noindent
Table \ref{tab:sprint_1_goals} shows a list of goals for the first sprint. The goals in \ref{tab:sprint_1_goals} is prioritized from top to bottom according to their importance. To establish a starting point, we need to understand what the previous semesters have been developing. When we have reviewed legacy code, we conclude which parts of aSTEP-2019 we need to redesign or develop. Furthermore, we agree upon the initial design of \gls{ui} and code documentation with the other groups of the aSTEP-2020 project. Finally, we conduct a risk analysis of our project.
