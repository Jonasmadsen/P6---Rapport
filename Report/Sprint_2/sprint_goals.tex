\section{Sprint Goals}
The sprint goal for the second sprint is to start designing and implementing the service. Before we design and implement our service, we analyze related work, by looking at some models used for anomaly detection. Some of these \gls{nn} models utilize methods for anomaly detection in ways irrelevant to our use case. However, analyzing these use cases help us understand the concepts and methods more in detail. After our analysis, we start the design of our service as a whole by designing the method of which to expose our service to \gls{astep}. \newline

\bgroup
\def\arraystretch{1.8}
\begin{table}[htbp]
    \centering
    \begin{tabular}{| m{7cm} |}
        \hline
        \multicolumn{1}{|>{\centering\arraybackslash}m{70mm}|}{\textbf{Goals}} \\
        \hline
        Review related work for anomaly detection. \\
        \hline
        Understand \glspl{nn} used for anomaly detection, including autoencoders,\gls{vae} and \gls{rnn}. \\
        \hline
        Design our \gls{astep} service. \\
        \hline
    \end{tabular}
    \caption{The second sprint goals.}
    \label{tab:sprint_2_goals}
\end{table}
\egroup

\noindent
Table \ref{tab:sprint_2_goals} shows a list of goals for the second sprint. The first goal is to review related work used for anomaly detection. Then we should be able to understand \glspl{nn} and their usage in anomaly detection. Understanding this will help us in the process of designing our \gls{nn} for anomaly detection. After understanding \glspl{nn}, we design the service we expose to \gls{astep}. Section \ref{sprint_2_review} evaluates these goals in the sprint review.
