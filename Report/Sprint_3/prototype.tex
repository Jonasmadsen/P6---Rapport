\section{Prototype}
We design the prototype to work a lot like the finished model, but for univariate data. We made this choice in order to see, if our model could detect anomalies on a simpler dataset before we train it on multivariate data.

\subsection{Implementation of The Prototype}
As mentioned, we are using \gls{tensorflow} and \gls{keras} for the implementation of the \gls{nn}. 
\begin{listing}[htbp]
\begin{minted}
[
frame=single,
framesep=3mm,
breaklines,
linenos=true,
xleftmargin=21pt,
fontsize=\footnotesize,
tabsize=4
]
{Python}
def make_encoder(batch_size, window_size, features, gru_dim, dense_dim):
    input = keras.Input(shape=(window_size, features), name='encoder_input')
    gru = keras.layers.GRU(gru_dim, return_sequences=True, name='encoder_gru')(input)
    dense = keras.layers.Dense(dense_dim, name='encoder_dense', activation='relu')(gru)
    z_mean = keras.layers.Dense(features, name='encoder_z_mean', activation='linear')(dense)
    z_log_var = keras.layers.Dense(features, name='encoder_z_log_var', activation='softplus')(dense)
    z_mean, z_log_var = KLDivergenceLayer()([z_mean, z_log_var])
    z_sample = keras.layers.Lambda(sampling, name='encoder_z_sample',
                                   output_shape=(None, window_size, features))([z_mean, z_log_var])

    encoder = keras.Model(input, z_sample, name='encoder')
    return encoder, input
\end{minted}
\caption{Encoder.}
\label{lst:encoder}
\end{listing}

\noindent

\subsection{Dataset For The Prototype}
Training the \gls{nn} requires a dataset to train on that contains anomalies. Yahoo has a collection of datasets \cite{yahoo_datasets}. One of these datasets (S3) is a collection of time-series data with varying anomalies. This dataset was created by Yahoo to benchmark anomaly detection algorithms. The dataset has both synthetic and real data. The synthetic data has a changing trend, noise, and seasonality, while the real data is data collected by Yahoo from various Yahoo services.

From the S3 dataset we use <synthetic\_85 to train the prototype. This is univariate data, and we use it to test if our \gls{nn} is able to detect anomalies.

\subsection{Training of The Prototype}