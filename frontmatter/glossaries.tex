\newglossaryentry{kubernetes}
{
    name=Kubernetes,
    description={A container orchestration system \cite{Kubernetes}},
    first={\textbf{Kubernetes}}
}
\newglossaryentry{docker}
{
    name=Docker,
    description={A platform to build, run and share applications with containers \cite{Docker}},
    first={\textbf{Docker}}
}
\newglossaryentry{gitlab}
{
    name=GitLab,
    description={A complete DevOps platform, delivered as a single application \cite{gitlab_what}},
    first={\textbf{GitLab}}
}
\newglossaryentry{endpoint}
{
    name=endpoint,
    description={A resource located at a URL, exposed by a service through an API},
    first={\textbf{endpoint}}
}
\newglossaryentry{graphql}
{
    name=GraphQL,
    description={A query language specification for interacting with an API},
    first={\textbf{GraphQL}}    
}
\newglossaryentry{postgres}
{
    name=PostgreSQL,
    description={A common object-relational database},
    first={\textbf{PostgreSQL}}
}

\newglossaryentry{backprop}
{
    name=backpropagation,
    description={A common method of training a neural network by computing the gradient of the loss function with respects to the weights of the network. Gradient descent or stochastic gradient descent is often used},
    first={\textbf{backpropagation}}
}

\newglossaryentry{graddesc}
{
    name=gradient descent,
    description={An iterative optimization algorithm for finding local minima of differentiable functions. A stochastic variation exists for stochastic functions},
    first={\textbf{gradient descent}}
}

\newglossaryentry{lossfunc}
{
    name=loss function,
    description={A loss function or cost function, is a function describing the difference between an expected target and an actual target. The higher the value, the more difference. Some notable loss functions include, Mean Squared Error and cross entropy},
    first={\textbf{loss function}}
}
\newglossaryentry{visiblelayer}
{
    name=visible layer,
    description={The layer(s) in a neural network that are observable. The input and output layers are usually the only visible layers},
    first={\textbf{visible layer}}
}
\newglossaryentry{hiddenlayer}
{
    name=hidden layer,
    description={The hidden layer(s) in a neural network are all the layers that are not visible. Usually, this means all the layers between the input and the output},
    first={\textbf{hidden layer}}
}
\newglossaryentry{omnianomaly}
{
    name=OmniAnomaly,
    description={A stochastic recurrent neural network, that novelly glues together a Recurrent Neural Network, with a Variational Autoencoder. See \cite{kdd} for more},
    first={\textbf{OmniAnomaly}}
}
\newglossaryentry{entity_level}
{
    name=entity level,
    description={When considering an entity for which some metrics are monitored, the entity level would be considering the whole entity instead of each individual metric. For multivariate time series data, entity level is considering all the time series variables instead of considering each as a univariate time series},
    first={\textbf{entity level}}
}
\newglossaryentry{tensorflow}
{
    name=TensorFlow,
    description={TensorFlow is an end-to-end open source platform for machine learning  \cite{TensorFlow}},
    first={\textbf{TensorFlow}}
}
\newglossaryentry{keras}
{
    name=Keras,
    description={Keras is a high-level neural networks API, written in Python and capable of running on top of TensorFlow, CNTK, or Theano \cite{Keras}},
    first={\textbf{Keras}}
}
\newglossaryentry{pytorch}
{
    name=PyTorch,
    description={PyTorch is an open source machine learning framework that accelerates the path from research prototyping to production deployment \cite{PyTorch}},
    first={\textbf{PyTorch}}
}
\newglossaryentry{dl4j}
{
    name=Deeplearning4j,
    description={Deeplearning4j is an open-source, distributed, deep learning library for the JVM \cite{dl4j}},
    first={\textbf{Deeplearning4j}}
}

\newglossaryentry{scikit-learn}
{
    name=scikit-learn,
    description={scikit-learn is a Python module that can integrate a range of different machine learning algorithms for supervised and unsupervised problems \cite{scikit-learn}},
    first={\textbf{scikit-learn}}
}
\newglossaryentry{f1score}
{
    name=$F_1$-score,
    description={The harmonic mean of precision and recall},
    first={$\mathbf{F_1}$\textbf{-score}}
}
\newglossaryentry{precision}
{
    name=Precision,
    description={Fraction of relevant instances among retrieved instances},
    first={\textbf{Precision}}
}
\newglossaryentry{recall}
{
    name=Recall,
    description={Fraction of total amount of relevant instances that were retrieved},
    first={\textbf{Recall}}
}
\newglossaryentry{iso}
{
    name=Isolation Forest,
    description={Isolation Forest is an unsupervised learning algorithm used to detect anomalies by isolating areas and comparing to the average \cite{isolation_forest}},
    first={\textbf{Isolation Forest}}
}
\newglossaryentry{lof}
{
    name=Local Outlier Factor,
    description={Local Outlier Factor is an unsupervised learning algorithm used to detect anomalies by measuring the local deviation of density of a given sample with respect to its k-nearest neighbors \cite{lof}},
    first={\textbf{Local Outlier Factor}}
}
\newglossaryentry{slwi}
{
    name=sliding windows,
    description={For array a = [a, b, c, d, e], then a sliding window of 3 runs over an underlying collection like: [a, b, c], [b, c, d], [c, d, e]},
    first={\textbf{sliding windows}}
}

\newacronym{astep}{aSTEP}{aau's Spatio TEmporal data analytics Platform}
\newacronym{rfc}{RFC}{Request for Comments, previously Request for Changes}
\newacronym{api}{API}{Application Programming Interface}
\newacronym{url}{URL}{Uniform Resource Locator}
\newacronym{http}{HTTP}{Hypertext Transfer Protocol}
\newacronym{cicd}{CI/CD}{Continuous Integration/Continuous Deployment}
\newacronym{rmmm}{RMMM}{Risk Mitigation, Monitoring and Management Plan}
\newacronym{nn}{NN}{Neural Network}
\newacronym{srnn}{S-RNN}{Stochastic Recurrent Neural Network}
\newacronym{vae}{VAE}{Variational Autoencoder}
\newacronym{rnn}{RNN}{Recurrent Neural Network}
\newacronym{lstm}{LSTM}{Long Short Term Memory}
\newacronym{gru}{GRU}{Gated Recurrent Unit}
\newacronym{kldiv}{KL divergence}{Kullback-Leibler divergence}
\newacronym{vrnn}{VRNN}{Variational Recurrent Neural Network}
\newacronym{vrae}{VRAE}{Variational Recurrent Autoencoder}
\newacronym{ui}{UI}{User Interface}
\newacronym{rvae}{RVAE}{Recurrent Variational Autoencoder}
\newacronym{its}{ITS}{AAU IT Services}