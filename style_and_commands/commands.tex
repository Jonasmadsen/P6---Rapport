% Project related.
  % Project information.
  \newcommand{\projecttitle}{ANOM: A(nother) step forward for aSTEP}
  \newcommand{\projectsubtitle}{Anomaly detection with a Neural network On Multivariate time series data}
 
  \newcommand{\projecttheme}{Developing Complex Software Systems}
  \newcommand{\projekttema}{Indlejret Systemer}
  \newcommand{\projectperiod}{Spring semester 2020}
  \newcommand{\projektperiode}{Efterårssemester 2019}
  %\newcommand{\projectcopies}{}
  \newcommand{\projectcompletion}{May 28th 2020}
  \newcommand{\projektaflevering}{19. december 2019}

  % Names.
  \newcommand{\groupname}{SW604F20}
  \newcommand{\supervisor}{Tung Kieu}
  \newcommand{\groupmembersbyfirstname}[0]{%
    Jeppe Andreas Arnfelt \newline
    Jonas Madsen \newline
    Kristian Otte \newline
    Kristian Simoni Vestermark \newline
    Rasmus Møller Jensen \newline
  }

% Custom styling commands.
  % Monospaced text.
  \newcommand*\justify{%
    \fontdimen2\font=0.4em% interword space
    \fontdimen3\font=0.2em% interword stretch
    \fontdimen4\font=0.1em% interword shrink
    \fontdimen7\font=0.1em% extra space
    \hyphenchar\font=`\-% allowing hyphenation
  }
  \newcommand{\mono}[1]{\texttt{\justify {#1}}}

  % Code mono
  \newcommand{\code}[1]{\sethlcolor{clcodeshade}\hl{\texttt{\justify{#1}}}} % Yes, this line is identical to the one in \mono, however the soul-package is fragile, and cannot see into another command.
  \soulregister{\justify}{1}

  % Chapter introductions.
  \newcommand{\chapterintro}[1]{#1}

  % CD paths with icon.
  %\newcommand{\cdpath}[1]{%
  %  % Param1: path
  %  \hyperref[app:cd]{\raisebox{-0.28ex}{\includegraphics[height=0.85em]{cd}}\mono{/#1}}%
  %}

  % Margin text.
  \newcommand{\margintext}[1]{\marginline{\textsf{\footnotesize #1}}}

  % Todo.
  \newcommand{\todo}[1]{\fxnote{#1}}

  % Dummy line.
  \newcommand{\dummy}{\usnote{Write an abstract here}}

% Document structure.
  % Sprints.

  % Document organization.
  \newenvironment{documentorganization}
    {\vspace{.5cm}\noindent\textbf{Report Organization}\quad The remainder of this report is organized in the following fashion:\begin{itemize}}
    {\end{itemize}}

  % Chapter organization.
  \newenvironment{chapterorganization}
    {\vspace{.5cm}\noindent\textbf{Chapter Organization}\quad This chapter is organized in the following fashion:\begin{itemize}}
    {\end{itemize}}

   % Abbreviation.
  \newenvironment{abbreviations}
    {\vspace{.5cm}\noindent\textbf{Chapter Abbreviations}\quad This chapter introduces the following abbreviations:\begin{addmargin}[\leftmargin]{0em}\begin{multicols}{2}\begin{description}[noitemsep, style=sameline]}
    {\end{description}\end{multicols}\end{addmargin}}

    % Dates.
    %\newenvironment{dates}
    %{\vspace{.5cm}\noindent\textbf{Project Dates}\quad \dummy:\begin{addmargin}[\leftmargin]{0em}\begin{multicols}{2}\begin{description}[noitemsep, style=sameline]}
    %{\end{description}\end{multicols}\end{addmargin}}

% Figures.
  % Normal figure.
  \newcommand{\fig}[3]{
    \begin{figure}[tbp]
      \centering
      %\rule{\textwidth}{0.005in}
      \includegraphics[width=0.75\textwidth]{#1}
      \caption[#2]{#3}\label{fig:#1}
      %\rule{\textwidth}{0.005in}
    \end{figure}
  }

  % Scaled figure.
  \newcommand{\figscaled}[4]{
    \begin{figure}[tbp]
      \centering
      \includegraphics[scale=#4]{#1}
      \caption[#2]{#3}\label{fig:#1}
    \end{figure}
  }

  % Figure with custom width.
  \newcommand{\figcustomwidth}[4]{
    \begin{figure}[tbp]
      \centering
      \includegraphics[width=#4]{#1}
      \caption[#2]{#3}\label{fig:#1}
    \end{figure}
  }

% References.
  \newcommand{\appendixref}[1]{\hyperref[#1]{Appendix \ref*{#1}}}
  \newcommand{\chapterref}[1]{\hyperref[#1]{Chapter \ref*{#1}}}
  \newcommand{\sectionref}[1]{\hyperref[#1]{Section \ref*{#1}}}
  \newcommand{\figureref}[1]{\hyperref[#1]{Figure \ref*{#1}}}
  \newcommand{\tableref}[1]{\hyperref[#1]{Table \ref*{#1}}}
  \newcommand{\listingref}[1]{\hyperref[#1]{Listing \ref*{#1}}}
  \newcommand{\onpage}[1]{on \hyperref[#1]{page \pageref*{#1}}}
  \newcommand{\equationref}[1]{\hyperref[#1]{Equation \ref*{#1}}}
  \newcommand{\algorithmref}[1]{\hyperref[#1]{Algorithm \ref*{#1}}}
  %\newcommand{\sprintref}[1]{\hyperref[#1]{Sprint \ref*{#1}}}

% Column types for tabulars.
%\newcolumntype{L}[1]{>{\raggedright\let\newline\\\arraybackslash\hspace{0pt}}m{#1}}
%\newcolumntype{R}[1]{>{\raggedleft\let\newline\\\arraybackslash\hspace{0pt}}m{#1}}

% Fix todo notes with externalize
\let\oldTodo\todo
\renewcommand{\todo}[1]{\tikzexternaldisable{}\oldTodo{#1}\tikzexternalenable{}}

% Fixme stuff.
%\newcommand{\fillin}[1]{\fxnote*[inline]{#1}{Fill in: }}

% Vector command.
%\newcommand{\omatrix}[1]{\ensuremath{\boldsymbol{#1}}}

% A better plus minus sign.
\makeatletter
\newcommand{\gpm}{\mathbin{\mathpalette\@gpm\relax}}
\newcommand{\@gpm}[2]{\ooalign{%
  \raisebox{.1\height}{$#1+$}\cr
  \smash{\raisebox{-.6\height}{$#1-$}}\cr}}
\makeatother


\definecolor{rosso}{RGB}{220,57,18}
\definecolor{giallo}{RGB}{255,153,0}
\definecolor{blu}{RGB}{102,140,217}
\definecolor{verde}{RGB}{16,150,24}
\definecolor{viola}{RGB}{153,0,153}

\makeatletter

\tikzstyle{chart}=[
    legend label/.style={font={\scriptsize},anchor=west,align=left},
    legend box/.style={rectangle, draw, minimum size=5pt},
    axis/.style={black,semithick,->},
    axis label/.style={anchor=east,font={\tiny}},
]

\tikzstyle{bar chart}=[
    chart,
    bar width/.code={
        \pgfmathparse{##1/2}
        \global\let\bar@w\pgfmathresult
    },
    bar/.style={very thick, draw=white},
    bar label/.style={font={\bf\small},anchor=north},
    bar value/.style={font={\footnotesize}},
    bar width=.75,
]

\tikzstyle{pie chart}=[
    chart,
    slice/.style={line cap=round, line join=round, very thick,draw=white},
    pie title/.style={font={\bf}},
    slice type/.style 2 args={
        ##1/.style={fill=##2},
        values of ##1/.style={}
    }
]

\pgfdeclarelayer{background}
\pgfdeclarelayer{foreground}
\pgfsetlayers{background,main,foreground}


\newcommand{\pie}[3][]{
    \begin{scope}[#1]
    \pgfmathsetmacro{\curA}{90}
    \pgfmathsetmacro{\r}{1}
    \def\c{(0,0)}
    \node[pie title] at (90:1.3) {#2};
    \foreach \v/\s in{#3}{
        \pgfmathsetmacro{\deltaA}{\v/100*360}
        \pgfmathsetmacro{\nextA}{\curA + \deltaA}
        \pgfmathsetmacro{\midA}{(\curA+\nextA)/2}

        \path[slice,\s] \c
            -- +(\curA:\r)
            arc (\curA:\nextA:\r)
            -- cycle;
        \pgfmathsetmacro{\d}{max((\deltaA * -(.5/50) + 1) , .5)}

        \begin{pgfonlayer}{foreground}
        \path \c -- node[pos=\d,pie values,values of \s]{$\v\%$} +(\midA:\r);
        \end{pgfonlayer}

        \global\let\curA\nextA
    }
    \end{scope}
}

\newcommand{\legend}[2][]{
    \begin{scope}[#1]
    \path
        \foreach \n/\s in {#2}
            {
                  ++(0,-10pt) node[\s,legend box] {} +(5pt,0) node[legend label] {\n}
            }
    ;
    \end{scope}
}

% each of the following has two versions
%   \crefname{environmentname}{singular}{plural}, to be used mid-sentence
%   \Crefname{environmentname}{singular}{plural}, to be used at the beginning of a sentence
\crefname{table}{table}{tables}
\Crefname{table}{Table}{Tables}
\crefname{figure}{figure}{figures}
\Crefname{figure}{Figure}{Figures}
\crefname{equation}{equation}{equations}
\Crefname{equation}{Equation}{Equations}


\let\URL\url
\makeatletter
\def\url#1{\@URL#1;;\@nil}
\def\@URL#1;#2;#3\@nil{%
  \URL{#1}\ifx\relax#2\relax\else; \URL{#2}\fi}
\makeatother

\newcommand\pro{\item[$+$]}
\newcommand\con{\item[$-$]}

% Vores egne
\let\oldsection\section
\renewcommand{\section}[1]{\FloatBarrier\oldsection{#1}}